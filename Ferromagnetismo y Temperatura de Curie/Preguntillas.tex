\documentclass[a4paper,10pt,fleqn,oneside]{article}
\usepackage{amsmath,amsfonts,amssymb}   %paquetes útiles, contienen símbolos
%\usepackage{enumitem}					%hace más facil manejar listas numeradas
\usepackage[osf,noBBpl]{mathpazo}		%tipo de letra 
\usepackage[utf8]{inputenc}				
\usepackage[T1]{fontenc}				%para poder usar tildes sin problemas
\usepackage[left=1cm,right=1cm,top=1cm,bottom=1cm,headheight=13.6pt]{geometry}
\usepackage{enumerate} % enumerados
\usepackage[utf8]{inputenc} % acentos sin codigo
\usepackage{graphicx}

\usepackage[utf8]{inputenc}
\usepackage[spanish]{babel}
\usepackage[poly,arrow,curve,matrix]{xy}
\usepackage{array}
\usepackage{leftidx}
\usepackage{mathrsfs}
\usepackage{graphicx}
\usepackage[ampersand]{easylist}
% Abreviaturas
\newcommand\CC{\mathbb{C}}
\newcommand\RR{\mathbb{R}}
\newcommand\QQ{\mathbb{Q}}
\newcommand\ZZ{\mathbb{Z}}
\newcommand\NN{\mathbb{N}}
\renewcommand\emptyset{\varnothing}
\newcommand\nbd\nobreakdash
\newcommand\Sum{\displaystyle\sum}
\renewcommand{\baselinestretch}{1.2}
\newcommand{\eps}{\varepsilon}
\providecommand{\modu}[1]{\vert\vert#1\vert\vert}
\begin{document}


%%\begin{center}
%%\begin{large}
%Cálculo avanzado - Primer cuatrimestre

%\bigskip
%Ejercicio para entregar

%\bigskip
%Bryan Malpartida
%\end{large}
%\end{center}
\bigskip
\noindent
\centering
Preguntas de Ferromagnetismo para el pelado botón


\begin{enumerate}[1.]
	\item \textbf{Explique los comportamientos ferromagneticos, paramagneticos y diamagneticos.}
		 
		 Estos fenómenos son producido por la falta de amor en el mundo 
	\item \textbf{¿Qué es una curva de histéresis? Grafique e indique la magnetización remanente y la coercitividad.}
	
		La curva de histéresis muestra la magnetización de un material en función a la intensidad del campo magnético que la induce.

\begin{figure}[h]
	\centering
	\includegraphics[scale=0.5]{histeresis.jpg}
	\caption{Hola, soy la curva de histéresis}
	\label{CH}
\end{figure}		
		
	\item \textbf{¿Qué es la temperatura de Curie? Explique que sucede con el material por debajo y por encima de la $\mathbf{T_c}$.}
	
	La temperatura de Curie de un material es la temperatura a partir de la cual el material pasa de un comportamiento ferromagnético a paramagnético.
	\item \textbf{¿Qué son los materiales magnéticamente duros y blandos?¿Para qué se usan?}
	
	Los materiales magnéticamente duros son aquellos que, una vez magnetizados, conservan dicha magnetización de manera permanente, mientras que los blandos tienden a perderla fácilmente.
	Los duros se pueden utilizar en motores eléctricos y generadores de corriente continua entre otros; y los blandos se pueden usar en transformadores, generadores, electroimanes etc.
	\item \textbf{¿Cómo funciona un transformador, un transformador diferencial y un auto-transformador?¿Para qué se usa cada uno?}
	\item \textbf{¿Cómo funciona un circuito integrador? Calcule la función de transferencia y la frecuencia de corte.}
	\item \textbf{¿Qué significa la integral de la curva de histéresis?}
	
	El área bajo la curva es proporcional a la energía perdida como calor durante la magnetización. 
	\item \textbf{¿Depende el comportamiento de hístéresis de la frecuencia?¿Y de la temperatura? Explique.}
	
\end{enumerate}


\end{document}
