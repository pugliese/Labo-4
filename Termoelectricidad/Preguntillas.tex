\documentclass[a4paper,10pt,fleqn,oneside]{article}
\usepackage{amsmath,amsfonts,amssymb}   %paquetes útiles, contienen símbolos
%\usepackage{enumitem}					%hace más facil manejar listas numeradas
\usepackage[osf,noBBpl]{mathpazo}		%tipo de letra 
\usepackage[utf8]{inputenc}				
\usepackage[T1]{fontenc}				%para poder usar tildes sin problemas
\usepackage[left=1cm,right=1cm,top=1cm,bottom=1cm,headheight=13.6pt]{geometry}
\usepackage{enumerate} % enumerados
\usepackage[utf8]{inputenc} % acentos sin codigo
\usepackage{graphicx}
\usepackage{multicol}

\usepackage[utf8]{inputenc}
\usepackage[spanish]{babel}
\usepackage[poly,arrow,curve,matrix]{xy}
\usepackage{array}
\usepackage{leftidx}
\usepackage{mathrsfs}
\usepackage{graphicx}
\usepackage[ampersand]{easylist}
% Abreviaturas
\newcommand\CC{\mathbb{C}}
\newcommand\RR{\mathbb{R}}
\newcommand\QQ{\mathbb{Q}}
\newcommand\ZZ{\mathbb{Z}}
\newcommand\NN{\mathbb{N}}
\renewcommand\emptyset{\varnothing}
\newcommand\nbd\nobreakdash
\newcommand\Sum{\displaystyle\sum}
\renewcommand{\baselinestretch}{1.2}
\newcommand{\eps}{\varepsilon}
\providecommand{\modu}[1]{\vert\vert#1\vert\vert}
\begin{document}


%%\begin{center}
%%\begin{large}
%Cálculo avanzado - Primer cuatrimestre

%\bigskip
%Ejercicio para entregar

%\bigskip
%Bryan Malpartida
%\end{large}
%\end{center}
\bigskip
\noindent
\centering
Preguntas de Termoelectricidad para el pelado botón


\begin{enumerate}[1.]
	\item \textbf{Explique a que se debe el efecto termoeléctrico en junturas bimetáicas}

	Al poner en contacto dos metales en contacto, la densidad de electrones libres en la superficie de contacto tienden a homogeneizarse localmente; por lo que debido a la diferencia de densidad electrónica, los metales la juntura se cargan opuestamente produciendo entonces una diferencia de potencial de la juntura bimetálica. Y como la difusión de electrones de un metal a otro depende de la temperatura, si se tiene otra juntura bimetálica a diferente temperatura existirá una diferencia de potencial entre ellas y si se completa el circuito fluirá corriente. 

	\item \textbf{Explique como funciona un dispositivo termoeléctrico. Uniones bimetálicas en serie. Celdas termoeléctrica en serie}
	
	\item \textbf{¿Cuáles son los factores que limitan la eficiencia de un dispositivo termoeléctrico?}
	
	Calentamiento por efecto Joule y conducción térmica que transmite calor desde la cara caliente a la fría.
	
	\item \textbf{¿Existe una corriente máxima que se pueda forzar a través de un dispositivo termoeléctrico? ¿Por qué?}
	
	\item \textbf{¿Cómo se compara la eficiencia de un enfriador termoeléctrico contra un refrigerador tradicional?}
	
	Los dispositivos termoeléctricos tiene una eficiencia entre $5-10\%$ de un refrigerador ideal, mientras que los sistemas tradicionales (por ciclo de compresión) entre $40-50\%$.
	
	\item \textbf{¿Dónde encuentran utilidad los dispositivos termoeléctricos? ¿Por qué?}
	
	Se pueden utilizar como piezas refrigeradoras en las CPU. También, algunas compañías de automóviles,
usan generadores termoeléctricos alimentados por perdidas de calor.
	\item \textbf{¿Puede usarse dispositivos termoeléctricos para generar electricidad? ¿Como es su eficiencia?}
	
	Según internet (y por la respuesta anterior), sí. No sé cual es su eficiencia.
\end{enumerate}


\end{document}
