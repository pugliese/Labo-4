\documentclass[a4paper,10pt,fleqn,oneside]{article}
\usepackage{amsmath,amsfonts,amssymb}   %paquetes útiles, contienen símbolos
%\usepackage{enumitem}					%hace más facil manejar listas numeradas
\usepackage[osf,noBBpl]{mathpazo}		%tipo de letra 
\usepackage[utf8]{inputenc}				
\usepackage[T1]{fontenc}				%para poder usar tildes sin problemas
\usepackage[left=1cm,right=1cm,top=1cm,bottom=1cm,headheight=13.6pt]{geometry}
\usepackage{enumerate} % enumerados
\usepackage[utf8]{inputenc} % acentos sin codigo
\usepackage{graphicx}
\usepackage{multicol}

\usepackage[utf8]{inputenc}
\usepackage[spanish]{babel}
\usepackage[poly,arrow,curve,matrix]{xy}
\usepackage{array}
\usepackage{leftidx}
\usepackage{mathrsfs}
\usepackage{graphicx}
\usepackage[ampersand]{easylist}
% Abreviaturas
\newcommand\CC{\mathbb{C}}
\newcommand\RR{\mathbb{R}}
\newcommand\QQ{\mathbb{Q}}
\newcommand\ZZ{\mathbb{Z}}
\newcommand\NN{\mathbb{N}}
\renewcommand\emptyset{\varnothing}
\newcommand\nbd\nobreakdash
\newcommand\Sum{\displaystyle\sum}
\renewcommand{\baselinestretch}{1.2}
\newcommand{\eps}{\varepsilon}
\providecommand{\modu}[1]{\vert\vert#1\vert\vert}
\begin{document}


%%\begin{center}
%%\begin{large}
%Cálculo avanzado - Primer cuatrimestre

%\bigskip
%Ejercicio para entregar

%\bigskip
%Bryan Malpartida
%\end{large}
%\end{center}
\bigskip
\noindent
\centering
Preguntas de Termoelectricidad para el pelado botón


\begin{enumerate}[1.]
	\item \textbf{Explique a que se debe el efecto termoeléctrico en junturas bimetáicas}

	El efecto termoeléctrico se debe a que la densidad de electrones libres varía de un metal a otro y con la temperatura. Por lo tanto, cuando se ponen en contacto dos metales distintos y/o a distinta temperatura, la densidad electrónica tiende a homogeneizarse, cargandolos opuestamente y produciendo una diferencia de potencial.

	\item \textbf{Explique como funciona un dispositivo termoeléctrico. Uniones bimetálicas en serie. Celdas termoeléctrica en serie}
	
	Un dispositivo termoeléctrico consta de dos metales distintos puestos en contacto (semiconductores N y P para una celda de Peltier). Si estos metales se encuentran a distinta temperatura y se cierra el circuito, la diferencia de potencial generará una corriente. Inversamente, si se hace circular una corriente, se producirá un gradiente de temperatura entre los metales. 
	
	Las celdas de peltier utilizadas para enfriar se ubican termicamente en paralelo y eléctricamente en serie, de forma que todas transmitan el calor en la misma dirección.
	
	\item \textbf{¿Cuáles son los factores que limitan la eficiencia de un dispositivo termoeléctrico?}
	
	Calentamiento por efecto Joule y conducción térmica que transmite calor desde la cara caliente a la fría.
	
	\item \textbf{¿Existe una corriente máxima que se pueda forzar a través de un dispositivo termoeléctrico? ¿Por qué?}
	
	\item \textbf{¿Cómo se compara la eficiencia de un enfriador termoeléctrico contra un refrigerador tradicional?}
	
	Los dispositivos termoeléctricos tiene una eficiencia entre $5-10\%$ de un refrigerador ideal, mientras que los sistemas tradicionales (por ciclo de compresión) entre $40-50\%$.
	
	\item \textbf{¿Dónde encuentran utilidad los dispositivos termoeléctricos? ¿Por qué?}
	
	Se pueden utilizar como piezas refrigeradoras en las CPU. También, algunas compañías de automóviles,
usan generadores termoeléctricos alimentados por perdidas de calor durante la combustión.
	\item \textbf{¿Puede usarse dispositivos termoeléctricos para generar electricidad? ¿Como es su eficiencia?}
	
	Pueden utilizarse para generar electricidad, dado que el proceso físico es el mismo. Una diferencia de temperatura entre las caras produce una diferencia de potencial y, al cerrar el circuito, una corriente. 
	
	 El rendimiento de un par P-N usado para generar electricidad vendrá dado por la potencia eléctrica útil consumida por una resistencia de carga R con un flujo térmico atravesando el material: \[ \eta = \frac{P}{\dot{Q}} = \frac{I\alpha_{AB}\Delta T+I^2R}{\alpha_{AB}IT_c+K\Delta T-\frac{1}{2}RI^2} \] 
	
	
\end{enumerate}


\end{document}
